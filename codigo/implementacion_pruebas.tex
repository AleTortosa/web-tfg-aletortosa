% Contenidos del capítulo.
% Las secciones presentadas son orientativas y no representan
% necesariamente la organización que debe tener este capítulo.
Las secciones presentadas son orientativas y no representan
necesariamente la organización que debe tener este capítulo.
\section{Implementación}
%Presentar cómo se ha organizado el desarrollo de los proyectos
%(capturas del IDE), trozos de código relevantes, cómo han quedado
%implementadas las interfaces gráficas de usuario, etc.

La aplicación desarrollada para el Trabajo de Fin de Grado es una plataforma web que permite la gestión y consulta de reseñas de lugares turísticos. El proyecto se compone de dos partes principales: el \textbf{frontend}, realizado con HTML y CSS para la estructura y el diseño visual, y el \textbf{backend}, implementado con el framework \textbf{Django} en Python, encargado de la lógica de negocio, la gestión de usuarios y la interacción con la base de datos.

El desarrollo se ha realizado siguiendo buenas prácticas de organización y modularidad, separando claramente la lógica de presentación (HTML/CSS) de la lógica de negocio y persistencia (Django). Esto ha permitido crear una web moderna, funcional y fácilmente mantenible.

El desarrollo de la aplicación se ha basado en el patrón \textbf{Modelo-Vista-Controlador} (MVC), que en Django se traduce en la organización en \textbf{Modelos}, \textbf{Vistas} y \textbf{Templates}. Esta estructura facilita la mantenibilidad y escalabilidad del proyecto.

\subsection{Flujo de trabajo y organización}

El proceso típico para añadir una nueva funcionalidad en la web ha sido:

\begin{enumerate}
    \item \textbf{Definir el modelo}: Se crea o modifica el modelo correspondiente en \texttt{models.py}, añadiendo los campos y relaciones necesarios.
    \item \textbf{Migrar la base de datos}: Se ejecutan las migraciones para reflejar los cambios en la base de datos.
    \item \textbf{Crear la vista}: Se implementa la lógica en \texttt{views.py}, donde se procesan las peticiones, se consulta la base de datos y se preparan los datos para la plantilla.
    \item \textbf{Configurar la URL}: Se añade la ruta correspondiente en \texttt{urls.py} para que la funcionalidad sea accesible desde la web.
    \item \textbf{Diseñar la plantilla}: Se crea o modifica el archivo HTML en la carpeta \texttt{templates/}, utilizando el sistema de plantillas de Django y extendiendo \texttt{base.html} para mantener la coherencia visual.
    \item \textbf{Gestionar en el admin}: Si la funcionalidad requiere gestión interna, se registra el modelo en \texttt{admin.py} para poder administrarlo desde el panel de Django.
\end{enumerate}

\subsection{Modelo}

El modelo representa la estructura de los datos. Por ejemplo, para las reseñas:

\begin{lstlisting}[language=Python, caption={Modelo de Reseña}]
class Resena(models.Model):
    usuario = models.ForeignKey('Usuario', on_delete=models.CASCADE)
    evento = models.ForeignKey(Evento, on_delete=models.CASCADE)
    descripcion = models.TextField()
    puntuacion = models.PositiveSmallIntegerField(validators=[MinValueValidator(0), MaxValueValidator(5)])
    fecha = models.DateTimeField(auto_now_add=True)
\end{lstlisting}

Cada vez que se añade o modifica un modelo, se ejecuta \texttt{python manage.py makemigrations} y \texttt{python manage.py migrate} para actualizar la base de datos.

\subsection{Vista}

Las vistas gestionan la lógica de negocio y la interacción con los modelos. Por ejemplo, para mostrar el detalle de un evento y calcular la puntuación media de sus reseñas:

\begin{lstlisting}[language=Python, caption={Vista de detalle de evento}]
def detalle_evento_view(request, evento_id):
    evento = get_object_or_404(Evento, id=evento_id)
    reseñas = evento.resena_set.all()
    puntuacion_media = reseñas.aggregate(Avg('puntuacion'))['puntuacion__avg']
    return render(request, "detalle_evento.html", {
        "evento": evento,
        "reseñas": reseñas,
        "puntuacion_media": puntuacion_media,
    })
\end{lstlisting}

\subsection{URLs}

Las URLs conectan las vistas con las rutas accesibles desde el navegador. Por ejemplo:

\begin{lstlisting}[language=Python, caption={Ruta para el detalle de evento}]
path('evento/<int:evento_id>/', views.detalle_evento_view, name='detalle_evento'),
\end{lstlisting}

\subsection{Templates (HTML)}

Las plantillas definen la interfaz gráfica. Se ha creado una plantilla principal \texttt{base.html} que incluye el menú, el footer y los estilos globales. Todas las demás plantillas extienden esta base para mantener la coherencia visual.

\begin{lstlisting}[language=HTML, caption={Extensión de base.html}]

Detalle de evento

<!-- contenido específico de la página -->

\end{lstlisting}

Por ejemplo, la plantilla de reseñas muestra los botones y la lista de reseñas recientes:

\begin{lstlisting}[language=HTML, caption={Fragmento de plantilla de reseñas}]
<div class="resenas-botones">
    
        
            <a href="" class="btn-resena">Escribir reseña</a>
        
            <a class="btn-resena btn-disabled">Escribir reseña (verifica tu cuenta)</a>
        
        <a href="" class="btn-resena">Ver mis reseñas</a>
    
        <a class="btn-resena btn-disabled">Escribir reseña (inicia sesión)</a>
        <a href="" class="btn-resena">Inicia sesión para ver tus reseñas</a>
    
</div>
\end{lstlisting}

\subsection{Zona de administración}

El panel de administración de Django permite gestionar usuarios, eventos, reseñas y reportes. Se han personalizado los modelos en \texttt{admin.py} para mostrar información relevante y facilitar la gestión.

\begin{figure}[!htb]
    \centering
    \includegraphics[width=0.7\textwidth]{admin_panel.png}
    \caption{Panel de administración personalizado}
\end{figure}

\begin{lstlisting}[language=Python, caption={Registro de modelos en el admin}]
@admin.register(ReporteResena)
class ReporteResenaAdmin(admin.ModelAdmin):
    list_display = ('resena', 'reportado_por', 'fecha', 'revisado')
    actions = ['marcar_como_revisado', 'eliminar_resena']
\end{lstlisting}

\subsection{CSS y diseño visual}

El diseño visual se ha realizado con CSS personalizado, definiendo variables de color y clases para botones, tarjetas y listas. Esto permite mantener una identidad visual clara y moderna en toda la web.

\begin{figure}[!htb]
    \centering
    \includegraphics[width=0.7\textwidth]{detalle_evento.png}
    \caption{Ejemplo de interfaz gráfica de detalle de evento}
\end{figure}

---

\textbf{En resumen}, la implementación sigue el flujo MVC, con una estructura modular y una interfaz clara y moderna. El uso de Django ha permitido una gestión eficiente de usuarios, permisos y recursos, y la web es fácilmente extensible y mantenible.

La interfaz gráfica se ha diseñado para ser intuitiva y visualmente atractiva, siguiendo los colores corporativos definidos en el CSS del proyecto.
\section{Pruebas unitarias}
Descripción de las pruebas que se han llevado a cabo para comprobar que el código
desarrollado es correcto (JUnit, etc).
\section{Pruebas funcionales}
Descripción de las pruebas que se han llevado a cabo para comprobar que los casos de uso
identificados funcionan correctamente.
\section{Pruebas de rendimiento}
Descripción de las pruebas de estrés realizadas para comprobar los tiempos de
respuesta de la aplicación (según figuren en los requisitos).
\section{Pruebas de usabilidad}
Descripción de las pruebas que se han llevado a cabo con usuarios para determinar el
nivel de usabilidad de la aplicación (que se hayan recogido en los requisitos).
\section{Pruebas de seguridad}
Descripción de las pruebas realizadas para comprobar que se cumplen las
restricciones de autenticación y de autorización que se han descrito
en los requisitos.


\subsection{Requisitos funcionales}

\begin{itemize}    
    \item \textbf{RF1:} El sistema debe permitir a los usuarios registrarse e iniciar sesión con su usuario y contraseña.
    \item \textbf{RF2:} El sistema debe permitir a los usuarios buscar lugares turísticos por nombre o ubicación.
    \item \textbf{RF3:} El sistema debe permitir a los usuarios ver y aplicar filtros avanzados basados en las etiquetas asignadas a cada lugar turístico.
    \item \textbf{RF4:} El sistema debe permitir a los usuarios locales reseñar lugares turísticos de su ciudad.
    \item \textbf{RF5:} El sistema debe permitir a los usuarios locales editar o eliminar sus reseñas.
    \item \textbf{RF6:} El sistema debe permitir a los usuarios ver la información detallada de un lugar turístico.
    \item \textbf{RF7:} El sistema debe permitir a los usuarios locales agregar nuevas reseñas.
    \item \textbf{RF8:} El sistema debe permitir a los usuarios ver las reseñas de otros usuarios.
    \item \textbf{RF9:} El sistema debe permitir a los usuarios ver el perfil de otros usuarios.
    \item \textbf{RF10:} El sistema debe permitir a los usuarios reportar reseñas inapropiadas.
    \item \textbf{RF11:} El sistema debe permitir a los usuarios ver las reseñas más recientes.
    \item \textbf{RF12:} El sistema debe permitir a los usuarios ver la puntuación media de un lugar turístico.
    \item \textbf{RF13:} El sistema debe permitir a los usuarios ver la ubicación de un lugar turístico en un mapa.
    \item \textbf{RF14:} El sistema debe permitir a los locales agregar una o varias imagenes opcionales a su reseña.
    \item \textbf{RF15:} El sistema debe permitir a los moderadores gestionar las reseñas publicadas.
    \item \textbf{RF16:} El sistema debe permitir a los administradores gestionar los usuarios del sistema.
    \item \textbf{RF17:} El sistema debe permitir a los administradores gestionar los moderadores del sistema.
\end{itemize}